\documentclass[12pt]{article}
\usepackage[utf8]{inputenc}
\usepackage{float}
\usepackage{amsmath}


\usepackage[hmargin=3cm,vmargin=6.0cm]{geometry}
%\topmargin=0cm
\topmargin=-2cm
\addtolength{\textheight}{6.5cm}
\addtolength{\textwidth}{2.0cm}
%\setlength{\leftmargin}{-5cm}
\setlength{\oddsidemargin}{0.0cm}
\setlength{\evensidemargin}{0.0cm}

\newcommand{\HRule}{\rule{\linewidth}{1mm}}

%misc libraries goes here
\usepackage{tikz}
\usetikzlibrary{automata,positioning}

\begin{document}

\noindent
\HRule \\[3mm]
\begin{flushright}

                                         \LARGE \textbf{CENG 222}  \\[4mm]
                                         \Large Statistical Methods for Computer Engineering \\[4mm]
                                        \normalsize      Spring '2018-2019 \\
                                           \Large   Homework 1 \\
\end{flushright}
\HRule

\section*{Student Information }
%Write your full name and id number between the colon and newline
%Put one empty space character after colon and before newline
Full Name : Yavuz Selim Yesilyurt \\
Id Number : 2259166 

% Write your answers below the section tags
\section*{Answer 2.15}
Let $A$ is the event that there is an error in the first block and let $B$ is the event that there is an error in the second block and $C$ is the event that the computer program returns error. Given that $P\{A\}= 0.2$ and $P\{B\} = 0.3$ and the events $A$ and $B$ are independent. The probability that the computer program will return error is $P\{A \cup B\}$ and it can be found adding all the cases that can bring an error and subtracting their intersection, namely we need to add probability that first block contains an error whereas second one does not, probability that second block contains an error whereas first one does not and we need to subtract the probability that both of them contains error: 

\begin{align*}
P\{A \cup B\} &= P\{A\} + P\{B\} - P\{A \cap B\} \\
&= 0.2 + 0.3 - 0.2 \times 0.3 \\
&= 0.44
\end{align*}

Then by definition of conditional probability:

\begin{align*}
P\{A \cap B \ | \ A \cup B\} &= \frac{P\{(A \cap B) \cap (A \cup B) \}}{P\{A \cup B\}}  \\
&= \frac{P\{A \cap B \}}{P\{A \cup B\}}  \\
&= \frac{0.06}{0.44} \\
&= 0.1364
\end{align*}

\section*{Answer 2.20}
Let $U$ denotes the event that an athlete uses steroids and let $P$ denotes the test gives a positive result. Then the given probabilities can be rewritten as:
\begin{center}
$P\{P\ |\ U\} = 0.9$ and $P\{P\ |\ \bar{U}\} = 0.02$ and $P\{U\} = 0.05$
\end{center}
We are required to find $P\{U\ | \ \bar{P}\}$. From the Bayes's rule, we have:
\begin{center}
$P\{U\ | \ \bar{P}\} = \frac{P\{\bar{P}\ |\ U\}\times P\{U\}}{P\{\bar{P}\}}$
\end{center}
We know that, from the law of total probability, $P\{\bar{P}\}$ can be rewritten as:
\begin{center}
$P\{\bar{P}\ |\ U\}\times P\{U\} \times P\{\bar{P}\ |\ \bar{U}\}\times P\{\bar{U}\}$
\end{center} 
So: 
\begin{align*}
P\{U\ | \ \bar{P}\} &= \frac{P\{\bar{P}\ |\ U\}\times P\{U\}}{P\{\bar{P}\}}  \\\\
&= \frac{P\{\bar{P}\ |\ U\}\times P\{U\}}{P\{\bar{P}\ |\ U\}\times P\{U\} \times P\{\bar{P}\ |\ \bar{U}\}\times P\{\bar{U}\}}  \\\\
&= \frac{0.1 \times 0.05}{0.1 \times 0.05 + 0.98 \times 0.95} \\\\
&= 0.00534
\end{align*}

\section*{Answer 2.29}

Let $A$ be the event that the database contain the given keyword. Then $P\{A\}=\frac{5}{9}$ and event that the database does not contain the given keyword is $P\{\bar{A}\}=\frac{4}{9}$ \\

Probability that the keyword will be found in at least 2 of the first 4 searched databases can be found by examining all the favorable outcomes, adding them up and then dividing them to total number of outcomes, namely: \\

 = finding in exactly 2 of the first 4 searched databases + finding in exactly 3 of the first 4 searched databases + finding in exactly 4 of the first 4 searched databases $\div$ choosing 4 databases to search for word out of 9 databases \\
 
Namely:
 
\begin{align*}
&= \frac{C(5,2)\times C(4,2) + C(5,3) \times C(4,1) + C(5,4)}{C(9,4)} \\
&= \frac{105}{126} \\\\
&= 0.833 
\end{align*}

\end{document}
